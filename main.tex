\documentclass{article}
\usepackage{graphicx} % Required for inserting images

\title{How to Prove Mathematical Statements}
\author{Rodolfo Lopez}
\date{March 2023}

\begin{document}

\maketitle

\section{Introduction}

In mathematics, a statement is a proposition that is either true or false. Proofs are used to establish whether a statement is true or false. We will discuss four particularly important methods of proof: direct proof, proof by contrapositive, proof by contradiction, and proof by induction. In this exposition, we will discuss and provide examples of each of these methods. 

\section{Direct Proof}

A direct proof is arguably the simplest and most common method of proof. It involves showing that an if-then statement is true by providing a logical chain of reasoning from the premises to the conclusion. The basic structure of a direct proof is as follows:

\begin{enumerate}
\item State the hypothesis or premise of the statement to be proved.
\item State any definitions or axioms that are relevant.
\item Use logical reasoning to deduce the conclusion from the premises and definitions.
\item State the conclusion and conclude the proof.
\end{enumerate}

\noindent\textbf{Example 1.} Prove that if $a$ and $b$ are odd integers, then $a+b$ is even.

\noindent\textbf{Solution:}
\begin{enumerate}
\item Let $a$ and $b$ be odd integers.
\item By definition, an odd integer can be written as $2n+1$, where $n$ is an integer.
\item So, we can write $a=2n_1+1$ and $b=2n_2+1$ for some integers $n_1$ and $n_2$.
\item Therefore, $a+b=2n_1+1+2n_2+1=2(n_1+n_2+1)$, which is even by definition.
\item Hence, if $a$ and $b$ are odd integers, then $a+b$ is even.
\end{enumerate}

In this example, we used direct proof to show that the sum of two odd integers is even. We started by assuming that $a$ and $b$ are odd integers and used logical reasoning to show that $a+b$ is even.

\section{Proof by Contrapositive}

A proof by contrapositive is a proof that establishes the truth of an if-then statement by proving the truth of its contrapositive. The contrapositive of an if-then statement is formed by negating both the hypothesis and the conclusion and reversing the direction of the implication. The basic structure of a proof by contrapositive is as follows:

\begin{enumerate}
\item State the contrapositive of the statement to be proved.
\item Assume the negated conclusion is true.
\item Use direct proof to show that the negated hypothesis must also be true.
\end{enumerate}

\noindent\textbf{Example 2.} Prove that if $n$ is an integer and $n^2$ is even, then $n$ is even.

\noindent\textbf{Solution:}

\begin{enumerate}
\item The contrapositive of the statement is: If $n$ is an odd integer, then $n^2$ is odd.
\item Assume that $n$ is an odd integer.
\item By definition, an odd integer can be written as $2k+1$, where $k$ is an integer.
\item So, $n^2=(2k+1)^2=4k^2+4k+1=2(2k^2+2k)+1$ is odd
\item Hence, if $n$ is an odd integer, then $n^2$ is odd.
\item By proving the contrapositive, we have shown that if $n^2$ is even, then $n$ must be even.
\end{enumerate}

In this example, we used proof by contrapositive to show that if $n^2$ is even, then $n$ must be even. We started by assuming that $n$ is an odd integer, and then used direct proof to show that $n^2$ must be odd. This, in turn, proved the contrapositive of the original statement.

\section{Proof by Contradiction}

A proof by contradiction is a proof that establishes the truth of a statement by assuming its negation and then deriving a contradiction. The basic structure of a proof by contradiction is as follows:

\begin{enumerate}
\item Assume the negation of the statement to be proved.
\item Use logical reasoning to derive a contradiction.
\item Conclude that the original statement must be true.
\end{enumerate}

\noindent\textbf{Example 3.} Prove that $\sqrt{2}$ is irrational.

\noindent\textbf{Solution:}
\begin{enumerate}
\item Assume that $\sqrt{2}$ is rational, i.e., it can be expressed as a fraction in lowest terms, $\sqrt{2}=\frac{p}{q}$, where $p$ and $q$ are integers with no common factors.
\item Squaring both sides of the equation, we get $2=\frac{p^2}{q^2}$.
\item Multiplying both sides by $q^2$, we get $2q^2=p^2$.
\item Since $p^2$ is even, $p$ must be even. [by Example 2]
\item Let $p=2k$ for some integer $k$.
\item Substituting this into the previous equation, we get $2q^2=(2k)^2=4k^2$, which implies that $q^2=2k^2$.
\item Since $q^2$ is even, $q$ must also be even. [by Example 2]
\item But this contradicts our assumption that $p$ and $q$ have no common factors, since they are both even.
\item Therefore, our assumption that $\sqrt{2}$ is rational must be false, and $\sqrt{2}$ is irrational.
\end{enumerate}

In this example, we used proof by contradiction to show that $\sqrt{2}$ is irrational. We assumed that $\sqrt{2}$ is rational and then derived a contradiction by showing that $p$ and $q$ cannot both be even if they have no common factors. This proved that our initial assumption was false, and hence $\sqrt{2}$ must be irrational.

\section{Proof by Induction}

A proof by induction is a method of proof that establishes the truth of a statement for all natural numbers (or all integers greater than some fixed integer) by proving it for a base case and then showing that if the statement is true for some integer $n$, then it must also be true for $n+1$. The basic structure of a proof by induction is as follows:

\begin{enumerate}
\item Prove the statement for a base case (usually $n=1$ or $n=0$).
\item Assume that the statement is true for $n=k$.
\item Show that it must also be true for $n=k+1$.
\item Conclude that the statement is true for all natural numbers $n$.
\end{enumerate}

\noindent\textbf{Example 4.} Prove that for all natural numbers $n$, $1+2+3+\cdots+n=\frac{n(n+1)}{2}$.

\noindent\textbf{Solution:}
\begin{enumerate}
\item Base case: When $n=0$, the left-hand side of the equation is $1$, and the right-hand side is $\frac{1(1+1)}{2}=1$. Therefore, the statement is true for $n=1$.
\item Assume that the statement is true for $n=k$, i.e., $1+2+3+\cdots+k=\frac{k(k+1)}{2}$.
\item Consider the case when $n=k+1$. We have:
\begin{align*}
1+2+3+\cdots+k+(k+1) &= \frac{k(k+1)}{2}+(k+1) \
&= \frac{k(k+1)+2(k+1)}{2} \
&= \frac{(k+1)(k+2)}{2} \
&= \frac{(k+1)((k+1)+1)}{2}.
\end{align*}
\item This shows that if the statement is true for $n=k$, then it must also be true for $n=k+1$.
\item Therefore, by the principle of mathematical induction, the statement is true for all natural numbers $n$.
\end{enumerate}

In this example, we used proof by induction to show that the formula for the sum of the first $n$ natural numbers, $1+2+3+\cdots+n=\frac{n(n+1)}{2}$, is true for all natural numbers $n$. We first proved the base case for $n=1$, and then assumed that the formula is true for $n=k$. Using this assumption, we showed that the formula must also be true for $n=k+1$. This completed the induction step, and we concluded that the formula is true for all natural numbers $n$.

\end{document}
